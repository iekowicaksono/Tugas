%%%%%%%%%%%%%%%%%%%%%%%%%%%%%%%%%%%%%%%%%%%%%%%%%%%%%%%%%%%%%%%%%%%%%%%%%%%
%
% Template for a LaTex article in English.
%
%%%%%%%%%%%%%%%%%%%%%%%%%%%%%%%%%%%%%%%%%%%%%%%%%%%%%%%%%%%%%%%%%%%%%%%%%%%

\documentclass[a4, 12px]{article}

% AMS packages:
\usepackage{amsmath, amsthm, amsfonts}

% Theorems
%-----------------------------------------------------------------
\newtheorem{thm}{Theorem}[section]
\newtheorem{cor}[thm]{Corollary}
\newtheorem{lem}[thm]{Lemma}
\newtheorem{prop}[thm]{Proposition}
\theoremstyle{definition}
\newtheorem{defn}[thm]{Definition}
\theoremstyle{remark}
\newtheorem{rem}[thm]{Remark}

% Shortcuts.
% One can define new commands to shorten frequently used
% constructions. As an example, this defines the R and Z used
% for the real and integer numbers.
%-----------------------------------------------------------------
\def\RR{\mathbb{R}}
\def\ZZ{\mathbb{Z}}

% Similarly, one can define commands that take arguments. In this
% example we define a command for the absolute value.
% -----------------------------------------------------------------
\newcommand{\abs}[1]{\left\vert#1\right\vert}

% Operators
% New operators must defined as such to have them typeset
% correctly. As an example we define the Jacobian:
% -----------------------------------------------------------------
\DeclareMathOperator{\Jac}{Jac}

%-----------------------------------------------------------------
\title{Eksplorasi Hyperparameter CNN dan Neural Network}
\author{Imam Ekowicaksono -
  \small (NIM. 33220306)
  %\small E12345\\
  %\small Spain
}
\date{}

\begin{document}
\maketitle

%\abstract{This is a simple template for an article written in \LaTeX.}

\section{Pendahuluan}

Persoalan klasifikasi merupakan salah satu persoalan populer yang ada pada topik pembelajaran mesin (\textit{machine learning}). Persoalan klasifikasi ini dapat ditangani dengan menggunakan metode klasifikasi yang berbasis neural network. Metode klasifikasi ini dapat digunakan untuk mengklasifikasi suatu data yang berupa gambar. 

\subsection{Subsection}\label{sec:nothing}

More text.

\subsubsection{Subsubsection}\label{sec:nothing2}

More text.

% Bibliography
%-----------------------------------------------------------------
\begin{thebibliography}{99}

\bibitem{Cd94} Author, \emph{Title}, Journal/Editor, (year)

\end{thebibliography}

\end{document}
